\chapter*{Introduction}
\addcontentsline{toc}{chapter}{Introduction}
The Teichm\"{u}ller space, $\T(S)$, was mostly studied to understand complex analytic structures that a given surface $S$ can admit and maps between them. Teichm\"{u}ller had introduced a metric on $\T(S)$ which measured the optimal quasi-conformal constant of quasi-conformal homeomorphisms between two complex structures on $S$. A detailed review of this is given in \cite{primer}.\\

Thurston's motivation for introducing the asymmetric metric in \cite{thurston} was to give a geometric perspective on $\T(S)$. For compact surfaces with negative Euler characteristic there is a bijective correspondence between complex structures and hyperbolic structures. Thus by replacing quasi-conformal maps between complex structures with Lipschitz maps between hyperbolic structures, we obtain Thurston's asymmetric metric as described in chapter II. Constructing Lipschitz maps with the optimal Lipschitz constant between hyperbolic structures is equivalent to constructing geodesics for Thurston's metric. In \cite{thurston}, the construction of stretch rays is the first construction of geodesic rays on Thurston's metric. Since then many more constructions for geodesics in the asymmetric have been given, for example by Papadopolous, Yamada, Ther\'{e}t. See \cite{survey} and references there in for an overview of different constructions of geodesics in Thurston's Asymmetric metric. In section 2 of chapter II of this thesis key properties of Thurston's metric are proved in detail, filling all the gaps in the proofs presented in \cite{thurston}.\\

In \cite{luo}, Luo shows that by cutting up a surface with compact boundary $\Sigma$ into colored right-angled hexagons and assigning to each non-boundary edge $e$ of the hexagons the radius coordinate $z(e)$, the map from $\T(\Sigma)$ to the space of weighted filling arc systems, $|\mathscr{A}_{\text{fill}}(\Sigma,\pa \Sigma)|\times \R_{+}$,  defined as $Y\mapsto \sum_{e} z(e) e$ is a homeomorphism. More recently this result has been generalized using orthogeodesic foliations and spines of surfaces; which are described in chapter III. Given any hyperbolic surface with boundaries and crowns $Y$ the closest point projection to the boundary of $Y$ defines a measured foliation on $Y$ called the orthogeodesic foliation $ \mathcal{O}(Y,\pa Y)$. Transverse to the core of the spine there are special arcs called ``dual arcs" arcs to the spine which form a filling arc system on $Y$. In \cite{farre}, using the measure of the edges of the spine corresponding to the dual arcs as weights the map $\underline{A}:\T(\Sigma)\to |\mathscr{A}_{\text{fill}}(\Sigma,\pa \Sigma)|\times \R_{+}$ is defined, and is shown to be a homeomorphism. This map is actually equivalent to the map described by Luo when the surface has compact boundary.\\

Given a closed hyperbolic surface $X$ and a lamination $\lambda$ on $X$ Calderon and Farre define dilation rays based at $X$ with respect to $\lambda$ in $\T(S)$. This construction is defined in section 3 of chapter III of this dissertation. In their paper, Calderon and Farre ask whether these dilation rays are geodesics in Thurston's metric. In the final section I present a new result proving that whenever $\lambda$ is a pants decomposition of $X$ then dilation rays are geodesics in Thurston's metric (theorem \ref{thm:main}).\\

The key idea in the proof is to ``average" Thurston stretch maps along ``opposite" maximal laminations so that the dual arcs of the pants are fixed at the endpoints. To show that this average map fixes the endpoints of the dual arcs a new space $\T_\pa(Y)$ is constructed which forms an $\R^n-$principal bundle over $\T(\Sigma)$ where $n$ is the number of boundary components of $Y$. Then it is shown that there is a special section of this where the dual arcs are preserved and that the points of this section are marked by the average map.\\

The first chapter of this dissertation covers some basic hyperbolic geometry and highlights some important results and concepts. The second chapter covers some Teichmuller theory and Thurston metric. The final chapter sets up the background for Dilation rays and the final section is devoted to proving theorem \ref{thm:main}.
