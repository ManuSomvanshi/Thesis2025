\chapter{Hyperbolic Geometry}
%
This chapter is a quick review of Hyperbolic geometry and Hyperbolic surfaces. Most results mentioned here are well known and thus proofs are omitted. Most of the proofs can be found in \cite{katok, casson, buser}. 
\section{The Plane, Disk, and Hyperboloid Model}
This section introduces three models of the hyperbolic plane and some of it's basic properties. These three will be used in later sections based on convenience in proofs. 
\begin{definition}
  The Hyperbolic plane is the upper-half plane $\H = \{z\in \C\ |\ \Im(z)>0\}$ with the Riemannian metric $\dd s^2= \l(\dd x^2 + \dd y^2\r)/y^2$. 
\end{definition}
This means that in the tangent space $T_z\H$ we have the inner product
\begin{align}
  \langle \xi, \eta \rangle = \frac{\Re(\xi\cdot\overline{\eta})}{\Im(z)^2}. 
\end{align}
This inner product induces the norm $\|\cdot\|$ on $T_z\H$. Using this Riemannian metric we can define the distance on $\H$ between two points $z,w$ as follows:
\begin{align}
  d(z,w) = \inf_{\g} \int_\g \|\dot{\g}(t)\| \dd t = \inf_{\g} \int_\g \frac{|\dot{\g}(t)|}{\Im(\g(t))} \dd t.
\end{align}
where the infimum is over all curves $\g$ between $z$ and $w$. The group of Mobius transformations with real coefficients are isometries of $\H$. This group is generated by $z\mapsto z+a,\ z\mapsto \lambda z$, and $z\mapsto -1/z$. It is a straightforward computation to show that this group acts transitively on $\H$. Mobius maps with real coefficients form a group isomorphic to $\PSL$. Hence we view $\PSL$ as a subgroup of $\text{Isom}(\H)$. In fact, we know a much stronger result:
\begin{proposition}[see \cite{katok, casson}]
  The isometry group of $\H$ is generated by $\PSL$ along with the map $z\mapsto 1/\bar{z}$. The orientation preserving isometries are precisely the elements of $\PSL$.
\end{proposition}
In any metric space $X$, we define a curve $\g:[a,b]\to X$ to be a \textit{geodesic} if $\g$ is an isometry, i.e. $d(\g(t),\g(t')) = |t-t'|$ for all $t,t'\in [a,b]$. The geodesics of $\H$ can be completely classified:
\begin{proposition}[see \cite{katok, casson}]
  $\H$ is uniquely geodesic and the geodesics of $\H$ are semi-circles and lines which are perpendicular to the boundary $\bH$.
\end{proposition}
\begin{rem}
  Note that the boundary $\bH$ is the boundary of the upper half plane inside the Riemann sphere. Thus $\bH = \R\cup \{\infty\}$.
\end{rem}
Let $D(0,1)$ be the open unit disk in $\C$. There is a natural biholomorphism $\vp:\H\to D(0,1)$ given by $z\mapsto (z-i)/(z+i)$. Using this map we can pull-back the Riemannian metric of $\H$ to $D(0,1)$. This metric on $D(0,1)$ turns out to be
\begin{align}
  \dd s^2 = \frac{4|\dd z|^2}{(1-|z|^2)^2}
\end{align}
This is called the \textit{Poincare Disk model} of the hyperbolic plane. Using the biholomorphism $\vp$, we deduce that the geodesics in the disk with the above metric are diameters and circular arcs which are perpendicular to the boundary, which is the unit circle in $\C$. We denote this model by $\H$ as well, as in most cases it either will not matter which model we are referring to or it will be clear by context. The isometries in the disk model explicitly are given by maps of the form
\begin{align}
  z\mapsto e^{i\theta}\frac{z-a}{1-\bar{a}z},\ |a|<1
\end{align}
These form a group isomorphic to $\PSL$.

Consider $\R^3$ with the bilinear form $h(X,Y) = x_1y_1 + x_2y_2 - x_3y_3$ and define the set $H = \{X\in \R^3\ |\ h(X,X) = -1 \And x_3>0\}$. The bilinear form induces a Riemannian metric on $H$ given by $\dd s^2 = \dd x_1^2 + \dd x_2^2 -\dd x_3^2$. The space $H$ with this induced metric is isometric to $\H$. This can be seen explicitly via the map $f:D(0,1)\to H$
\begin{align}
  f(r,\theta) = \l(\frac{2r}{1-r^2}\cos(\theta), \frac{2r}{1-r^2}\sin(\theta), \frac{1+r^2}{1-r^2}\r)
\end{align}
Thus $H$ is just another model for $\H$ and is called the \textit{Hyperboloid model}. The isometries in this model are just matrices $A\in GL_3(\R)$ which preserve the bilinear form. This is the group $SO_0(2,1)$, which is isomorphic to $\PSL$. Explicitly these are generated by the matrices of the form (see \cite{buser}):
\begin{align}
  L_\sigma = \begin{pmatrix}\cos\sigma & -\sin\sigma & 0\\ \sin\sigma & \cos\sigma & 0\\ 0 & 0 &1\end{pmatrix} \And
  M_\rho = \begin{pmatrix}\cosh\rho & 0& \sinh\rho \\  0 & 1 &0\\ \sinh\rho & 0 & \cosh\rho \end{pmatrix}
\end{align}
Consider the point $p_0= (0,0,1)$ and the curve $\mu(t) = M_t(p_0)$, which by definition is a geodesic. This geodesic corresponds to the geodesic $ie^t$ in the plane model. 
%
\section{Area and Trigonometry}
A polygon in $\H$ is a subset $P\subset \H\cup\bH$ which is bounded by finitely many piece-wise by geodesics. Studying polygons gives a concrete picture of geometry in $\H$. The key aspects to understand are what is the area of a given polygon and how are the sides and angles of the polygons related. 
\begin{definition}
  The area measure induced by the Riemannian metric on the plane $\H$ is:
  \begin{align}
    \mu(S) = \int_S \frac{\dd x\dd y}{y^2}
  \end{align}
  where $S\subset \H$.
\end{definition}
This measure is preserved under the action of $\PSL$. The most important result regarding the area of polygons is the Gauss-Bonnet theorem.
\begin{theorem}[\cite{katok}]
  The area of a hyperbolic triangle $\Delta$ with angles $\a,\b,\g$ is $\mu(\Delta) = \pi-\a-\b-\g$.
\end{theorem}
An immediate corollary of this is that the area of any $n-$sided polygon $P$ with angle $(\a_i)_{i=1}^n$ is $\mu(P) = (n-2)\pi - \sum_i \a_i$.\\

Given a triangle $\Delta$ with side lengths $a,b,c$ and angles opposite to the sides $\a,\b,\g$, is it possible to deduce a relation between these quantities like in the case of Euclidean geometry? The answer is yes!

\begin{theorem}[Sine and Cosine Rule]
  Given a triangle $\Delta$ with side lengths $a,b,c$ and angles $\a,\b,\g$:
  \begin{align}
    &\cosh{c} = -\sinh{a}\sinh{b}\cos{\g} + \cosh{a}\cosh{b}\\
    &\cos{\g} = \sin{\a}\sin{\b}\cosh{c} - \cos{\a}\cos{\b}\\
    &\frac{\sinh{a}}{\sin{\a}} = \frac{\sinh{b}}{\sin{\b}} = \frac{\sinh{c}}{\sin{\g}}
  \end{align}
\end{theorem}
\begin{proof}[Proof from \cite{buser}]
  We work in the Hyperboloid model. Without loss of generality assume that the side $c$ is placed along the geodesic $\mu$ and the vertex with angle $\a$ is at $p_0$. Note that $M_\rho$ performs a translation along $\mu$ and $L_\sigma$ performs a rotation about $p_0$. From figure \ref{fig:triangle0} we get,
  \begin{align}
    L_{\pi-\a}M_b L_{\pi-\g} M_a L_{\pi-\b} M_c = I_3
  \end{align}
  Carrying out the matrix multiplication and rearranging we complete the proof.
  \begin{figure}[h]
    \centering
    \def\svgwidth{0.7\textwidth}
    \import{figures/Chapter1}{triangle.pdf_tex}
    \caption[Proof of sine and cosine laws.]{The translation $M_c$ brings the vertex with angle $\b$ to $p_0$ and $L_{\pi-\b}$ rotates it so that the length $a$ is along $\mu$. Repeating this, we eventually get the triangle back.}
    \label{fig:triangle0}
  \end{figure}
\end{proof}
The same technique can be applied to any polygon to get relations between the sides and angles. As right angled hexagons will be important later on I state the Sine and Cosine rules for that here.
\begin{theorem}[Sine and Cosine Rule for Hexagons]\label{thm:hexagon}
  Let $H$ be a hexagon with side lengths $a,\g, b, \a, c, \b$ (in that order). Then
  \begin{align}    
    &\cosh{c} = \sinh{a}\sinh{b}\cosh{\g} - \cosh{a}\cosh{b}\\
    &\coth{\a} \sinh{\g} = \cosh{\g}\cosh{b} - \coth{a}\sinh{b}\\
    &\frac{\sinh{a}}{\sinh{\a}} = \frac{\sinh{b}}{\sinh{\b}} = \frac{\sinh{c}}{\sinh{\g}}
  \end{align}
\end{theorem}
\section{$(X,G)$ structures and Hyperbolic Surfaces}\label{sec:gx}
Let $X$ be a differential $n-$manifold and let $G$ be a Lie group which acts analytically\footnote{This means that whenever $g_1,g_2\in G$ such that for some open set $U\subset X$, $g_1|_{U} = g_2|_{U}$ then $g_1=g_2$. The motivation for this comes from analytic continuation.} and transitively on $X$ via diffeomorphisms. A smooth map $f:U\to X$ where $U\subset X$ is said to be locally-$G$ if for all open sets $U_i\subset U$ there exists a $g_i\in G$ such that $g_i|_{U_i} = f|_{U_i}$.
\begin{definition}[$(X,G)-$structures as defined in \cite{goldman}]
   We say that an $n-$manifold $M$ has $(X,G)$ structure if there is an atlas $\{(U_\a, \vp_\a:U_\a\to X)\}$ of $M$ where $\vp_\a$ are homeomorphisms onto their images and the transition maps are locally-$G$ maps. 
\end{definition}
A map $f:M\to N$ between two $(X,G)-$manifolds is called a $(X,G)-$map if $\psi_\b\circ f\circ \vp_\a^{-1}$ is a locally-$G$ for coordinate patches $(U_\a,\vp_a)$ on $M$ and $(V_\b,\psi_\b)$ on $N$. It is easy to check that any locally-$G$ map $f$ has the unique extension property, i.e. there is a unique $g\in G$ such that $g|_{U} = f$.\\

Given a $(X,G)-$manifold $M$ there is a unique $(X,G)-$structure on the universal cover $p:\tilde{M}\to M$. Moreover, the Deck group $\pi_1(M)$ will act on $\tilde{M}$ by $(X,G)-$automorphisms. Fix a base point $p\in \tilde{M}$ and a coordinate neighborhood $(V,\psi)$ of $p$. Define an extension $\text{Dev}_{p,\psi}:\tilde{M}\to X$ of $\psi$ as follows:
\begin{enumerate}
  \item For any point $m\in \tilde{M}$ let $\g:[0,1]\to \tilde{M}$ be a path from $p$ to $m$. The path can be covered by finitely many coordinate charts $\{V_i, \psi_i\}$ such that $V_0 = V$ and $\psi_0 = \psi$ and there are intervals $(a_i,b_i)$ with
    \begin{align}
      0=a_0< a_1 < b_0 < a_2 < b_1 < a_3 \cdots < b_n = 1 
    \end{align}
    and $\{\g(t)\}_{a_i< t < b_i}\subset U_i$. See figure \ref{fig:dev}.
  \item By the unique extension property, there is a $g_i\in G$ such that $g_i\circ \psi_i = \psi_{i-1}$ on the open set $U_i\cap U_{i-1}$.
  \item Define $\text{Dev}_{p,\psi} = g_1\cdots g_n \psi_n(m)$. 
\end{enumerate}
The map $\text{Dev}_{p,\psi}:\tilde{M}\to X$ is well defined and unique extension of $\psi:V\to X$. For some other chart $(V',\psi')$ the extension will differ by the action of some unique $g$. For some $\gamma\in \pi_1(M)$ both $\text{Dev}$ and $\text{Dev}\g$ are developing maps extending different charts and thus it follows that there is some unique $g\in G$ such that $\text{Dev}\g = g\text{Dev}$. The map $\hol:\g\mapsto g$ is an injective homomorphism from $\pi_1(M)\to G$. A proof of these claims can be found in \cite{ratcliffe}. The map $\text{Dev}$ is called the \textit{developing map} and $\hol$ is called the \textit{holonomy}.\\

\begin{figure}
  \centering
  \def\svgwidth{0.7\textwidth}
  \import{figures/Chapter1}{drawing.pdf_tex}
  \caption[Developing map.]{The construction of developing map.}
  \label{fig:dev}
\end{figure}

The following theorem from \cite{goldman} summarizes the above discussion:
\begin{theorem}[Development Theorem]
  Let $M$ be an $(X,G)-$manifold and $p:\tilde{M}\to M$ be a universal cover of $M$. Then there exists a pair $(\text{Dev}:\tilde{M}\to X,\hol:\pi_1(M)\to G)$ such that for all $\gamma\in \pi_1(M)$ the diagram
  \[\begin{tikzcd}[sep=large]
	{\tilde{M}} & X \\
	{\tilde{M}} & X
	\arrow["{\text{Dev}}", from=1-1, to=1-2]
	\arrow["\gamma"', from=1-1, to=2-1]
	\arrow["{\text{hol}(\gamma)}", from=1-2, to=2-2]
	\arrow["{\text{Dev}}"', from=2-1, to=2-2]
\end{tikzcd}\]
commutes. Here $\text{Dev}$ is a developing map and the homomorphism $\hol$ is called the \textit{holonomy representation} of $\pi_1(M)$. Moreover, if there is another pair $(\text{Dev}',\hol')$ with the same properties, then $\text{Dev}' = g\text{Dev}$ and $\hol' = g\cdot \hol \cdot g^{-1}$ for some unique $g\in G$.
\end{theorem}

\begin{definition}
  A hyperbolic surface $M$ is a $(\H, \PSL)-$manifold. The $(X,G)$ maps in this case are just local isometries.
\end{definition}
From the development theorem, $\pi_1(M)$ can be viewed as a discrete subgroup of $\PSL$ using holonomy and we also have a developing map $\text{Dev}:\tilde{M}\to \H$. The local $\H$ structure gives the manifold $M$ a Riemannian metric, and geodesics in $M$ will locally be mapped to geodesics in $\H$. A hyperbolic surface $M$ is said to be \textit{complete} if every geodesic in $M$ can be extended indefinitely. 
\begin{theorem}
  A simply connected and complete hyperbolic surface $M$ is isometric to $\H$.
\end{theorem}
\begin{proof}[Proof from \cite{casson}]
  The proof involves constructing the map $\exp:\H\to M$ for a fixed $p\in M$ and a chart $(V,\psi)$ around $p$ follows: for any point $z\in\H$ let $\g$ be the geodesic from $\psi(p)$ to $z$, and pull back the geodesic arc $\g\cap \psi(V)$ to a geodesic $\g'$ in $M$. Define $\exp(z)$ to be the point on $\g'$ such that its distance from $p$ is $d(z,\psi(p))$.\\
  The inverse of this map is the developing map extending $\psi$. Since both of these maps are local isometries, it follows that they are isometries in this case. For details see \cite{casson}.
\end{proof}
Since any closed surface $X$ is complete it's universal cover is complete thus, by the above theorem, $\tilde{X}$ is isometric to $\H$. Rephrasing, any closed hyperbolic surface $X$ is isometric to $\H/\Gamma_X$ where $\Gamma_X$ is some discrete subgroup of $\PSL$. Motivated by this discussion the concept of fundamental domains can be defined.
\begin{definition}
  Let $\Gamma$ be a discrete subgroup of $\PSL$. A domain $F\subset \H$ is said to be a \textit{fundamental domain} of $\Gamma$ if
  \begin{enumerate}
    \item It tiles $\H$, i.e. $\bigcup_{\g\in \Gamma} \g F = \H$,
    \item $F^\circ \cap \g F^\circ = \emptyset$ for all $\g\in \Gamma$.
  \end{enumerate}
\end{definition}
Although the fundamental domain is not unique often one uses a \textit{Dirichlet Domain} construction for the fundamental domain of the group, see \cite{katok}. The Dirichlet domain of closed surfaces is a hyperbolic polygon with finitely many sides. The quotient $\H/\Gamma$ can be thought of as gluing the sides of the Dirichlet domain of $\Gamma$ in pairs. Moreover, these side pairings generate the group $\Gamma$ (\cite{katok}).\\ 

For the sake of later discussions define an affine manifold as follows.
\begin{definition}
  An affine manifold $M$ is a $(\R^n, \text{Aff}(\R^n))-$manifold. The $(X,G)-$maps in this case are called \textit{affine} maps.
\end{definition}
Here are two examples of affine manifolds which is all that is really needed for the purposes of this thesis.
\begin{exmp}
  Consider the positive real line $\R^+$ with the usual manifold structure. The map $\log:\R^+\to \R$ which maps $y\mapsto \log(y)$ is a homeomorphism with inverse being $x\mapsto e^x$. This map gives an affine manifold structure to $\R^+$ with local charts given by restrictions of $\log$. Using this structure it is easy to see that any affine map $f:\R^+\to \R^+$ is of the form $y\mapsto by^a$. With this structure $\R^+$ is affine isomorphic to $\R^+$.
\end{exmp}
\begin{exmp}
  Another example of an affine manifold is a circle $S^1$. Circle has a natural affine structure as it can be written as the quotient $\R/a\Z$. An affine isomorphism between two affine circles $f:\R/\Z\to \R/a\Z$ is such that the lift $\tilde{f}:\R\to \R$ is equivariant affine map, i.e. it's lift is of the form $\tilde{f}(x) = ax+b$ for any $b\in \R$ in this example. 
\end{exmp}
\section{Foliations, Geodesic Laminations, and Transverse Measures}
% foliation
The theory of foliations has many important applications in the study of Teichmuller spaces, see \cite{FLP}. In this section some basic facts about measured foliations and laminations are presented.
\begin{definition}
  A measured foliation $\mathscr{F}$ on a compact surface $S$ with singularities of order $k_1,\cdots, k_n \in \N$ at points $x_1,\cdots,x_n$ is given by an open cover $\{U_i\}$ of $M-\{x_1,\cdots, x_n\}$ and a non-vanishing, smooth, closed, and real valued one-form $\omega_i$ on each $U_i$ such that:
  \begin{enumerate} 
	\item $\omega_i = \pm\omega_j$ on $U_i\cap U_j$
	\item At each $x_i$ there is a local chart $(u,v):V\to \R^2$ such that $z=u+iv$ with $\omega_i = \Im(z^{k_i/2} \dd z)$ on $V\cap U_i$, for some branch of $z^{k_i/2}$.
  \end{enumerate}
  $\{(U_i,\omega_i)\}$ is called the atlas of the foliation.
\end{definition}
This definition has a nice geometrical picture. Suppose $p\in S-\{x_1,\cdots,x_n\}$ lies in $U_i$ and let $v_i:U\to \text{T}(U)$ be a vector field which is in the kernel of $\omega_i$. The unoriented flow lines of $v_i$ and $v_j$ in $U_i\cap U_j$ due to the compatibility condition of the one forms, and thus they join up to give ``lines" on the surface which are called the \textit{leaves} of the foliation $ \mathscr{F}$. By definition, the leaves are disjoint and cover the entire surface except for the singularities $x_i$ where $k_i+2$ leaves meet. The ``measure" part of measure foliation comes from the following: if $\a$ is an arc that is transverse to the flow lines of $v_i$ then the integral of $|\omega_i|$ on the arc $U_i\cap \a$ gives a measure $\mu_\a$ to the arc $\a$ which is well defined because of the compatibility condition 1. This measure is called the \textit{transverse measure of the foliation}. This measure has full support, meaning that every point in $S-\{x_1,\cdots,x_n\}$ is in the support of $\mu_\a$ for some $\a$.\\

The following are some examples of \textit{measured foliations}.
\begin{exmp}\label{ex:foliation1}
  Consider the foliation $ \mathscr{F}$ on $\R^2$ with singularity at $(0,0)$ and chart $(U= \R- (0,0), \omega = ydx+xdy)$. A vector field in the kernel of this one form is $x\partial_{x} - y\partial_y$, and thus the flow lines corresponding to this are $\Phi(t,(x,y)) = (xe^t, ye^{-t})$. In any neighborhood $V$ of the origin consider the natural chart inclusion chart. Then in $V\cap U$ we have $\omega = \Im(z\dd z)$, implying that the singularity is of order 2. Thus geometrically the foliation $ \mathscr{F}$ looks as in \ref{fig:foliation_example1}.  
\end{exmp}
\begin{figure}[h]
  \centering
  \begin{minipage}[c]{0.4\textwidth}
    \def\svgwidth{\textwidth}
    \import{figures/Chapter1}{drawing-1.pdf_tex}
    \caption[Foliation on $\R^2$]{The measured foliation $ \mathscr{F}$ on $\R^2$ constructed in example \ref{ex:foliation1}.}
    \label{fig:foliation_example1}
  \end{minipage}
  \hfill
  \begin{minipage}[c]{0.4\textwidth}
    \def\svgwidth{\textwidth}
    \import{figures/Chapter1}{foliation.pdf_tex}
    \caption[Foliation on $S_2$]{The measured foliation $ \mathscr{F}$ on $S_2$ constructed using a quadratic differential with 2 zeros of order $2$.}
    \label{fig:foliation_example2}
  \end{minipage}
\end{figure}
\begin{figure}[h]
  \centering
\end{figure}
\begin{exmp}\label{ex:foliation2}
  Let $M$ be any compact Riemann surface and $q$ be a holomorphic quadratic differential on $M$ which has zeros $x_1,\cdots, x_n$ of order $k_1,\cdots, k_n$. This means that in local charts $q$ takes the form $f(z)\dd z^2$ where $f$ is holomorphic. Consider $M - \{x_1,\cdots, x_n\}$ and an cover $\{U_i\}$ where $U_i$ is simply connected. In each cover define $\omega_i = \Im(\sqrt{q})$, where a choice of the branch is made in the charts. By definition, $\omega_i$ are compatible. For any neighborhood $V$ of $x_i$ there are charts given by integrating $\omega_i$ such that locally $q$ is $z^{k_1}\dd z^2$. Thus both conditions are satisfied and $\{U_i, \omega_i\}$ is a measured foliation $ \mathscr{F}$ on $M$.\\

  This is an important example since it was shown in \cite{hubbard} that almost all measured foliations on Riemann surfaces are of the above type, i.e. they are determined by the imaginary foliation of the quadratic differential.
\end{exmp}

Foliations are very closely related with the topology of the space. Using the Poincare-Hopf formula, the following result for foliations can be proven (see \cite{FLP}).
\begin{theorem}[Euler-Poincare Formula]
  If $ \mathscr{F}$ is a measured foliation with singularities $x_1,\cdots, x_n$ of order $k_1,\cdots, k_n$ on a compact surface $S$ such that no singularity lies on the boundary of $S$. Then
  \begin{align}
    2\chi(S) = -\sum_{i=1}^n k_i
  \end{align}
\end{theorem}
This significantly restricts the measured foliations a surface $S$ can admit. For instance, in $S_2$ there can only be three types of measured foliations: with singularities of order $(1,1,1,1),\ (1,1,2),\ (2,2)$. Figure \ref{fig:foliation_example2} is a foliation of the type $(2,2)$.\\

Since small perturbations of the leaves of a foliation are not geometrically relevant and so an equivalence relation is introduced as follows. On a surface $S$ the foliations $ \mathscr{F}_1 \sim \mathscr{F}_2$ if
\begin{itemize}
  \item there is an isotopy of $S$ which transforms $ \mathscr{F}_1$ to $ \mathscr{F}_2$
  \item $ \mathscr{F}_1$ can be transformed into $ \mathscr{F}_2$ by finitely many transformations of the following type:
\end{itemize}
\begin{figure}[h]
  \centering
  \def\svgwidth{0.5\textwidth}
  \import{figures/Chapter1}{whitehrad.pdf_tex}
  \caption{Whitehead moves.}
  \label{fig:whitehead}
\end{figure}
Transformations as in \ref{fig:whitehead} are called \textit{Whitehead moves}. The collection of all measured foliation on $S$ upto this equivalence is denoted $ \mathcal{MF}(S)$, the space of measured foliations.\\

Working with measured foliations can get quite complicated as they are often thought of as the equivalence classes in $ \mathcal{MF}(S)$. In the case of hyperbolic surfaces, there is a very simple and well understood geometric object called \textit{geodesic lamination} which encodes equivalent information about our surface $S$ as a foliation. 
\begin{definition}
  A geodesic lamination $\lambda$ on a hyperbolic surface $M$ is a collection of pairwise disjoint complete simple geodesics $\{\g_\a\}$ such that the $\cup_{\a}\g_\a$ is a closed set in $M$.
\end{definition}
Examples of geodesic laminations are closure of any disjoint collection of geodesics in the surface. This forms a lamination since under the Hausdorff metric geodesics converge to geodesics (see \cite{casson}). The leaf $\g$ is said to be \textit{isolated} if at each point $x\in \g$ there is a small neighborhood $U$ such that $(U,U\cap \lambda)$ is homeomorphic to $(\text{Disk}, \text{diameter})$. A leaf $\g$ is called \textit{proper} if for every point $x\in\g$ there is a neighborhood $U$ such that $U\cap\g$ has only one connected component. It is easy to see that every isolated leaf is a proper leaf. A \textit{minimal sub-lamination} $\lambda'\subset \lambda$ is a lamination which does not contain any sub-lamination. A geodesic lamination $\lambda$ is said to be maximal if there is no other geodesic lamination on $M$ containing $\lambda$. 
\begin{rem}
  Geodesic laminations on hyperbolic surfaces are measure zero subsets of the surface \cite{casson}. This can be used to see that every maximal lamination cuts the surface $M$ into hyperbolic triangles: if not then one can always add another simple geodesic on the component which is not a triangle. 
\end{rem}
\begin{figure}[t]\label{fig:laminations}
  \centering
  \def\svgwidth{0.7\textwidth}
  \import{figures/Chapter1}{laminations.pdf_tex}
  \caption[Example of Laminations]{The first image is a measured geodesic lamination with the transverse measure just being the counting measure. The second image shows a lamination which is the closure of a geodesic spiralling onto two compact geodesics. Here the compact geodesics are proper but not isolated.}
\end{figure}
\begin{definition}
  A transverse measure on a geodesic lamination $\lambda$ is an assignment $\mu$ which assigns to each transverse arc $\a$ a Radon measure $\mu_\a$ such that:
  \begin{enumerate}
    \item if $\b\subset \a$ then $\mu_\b = \mu_\a|_{\b}$.
    \item if $\a$ is homotopic to $\a'$ relative to $\lambda$ then the measure of any measurable subset of $\a$ is invariant under the homotopy. 
  \end{enumerate}
  We say that the transverse measure has full support when $\text{supp}(\mu_\a) = \lambda\cap \a$ for all transverse arcs $\a$. 
\end{definition}
\begin{rem}
  A leaf $\g$ is said to be in the support of the transverse measure if $\g \subset \bigcup_{\a} \text{supp} (\mu_\a)$. In the case when the measure is of full support every leaf is in the lamination. 
\end{rem}
A theorem of Thurston (\cite{thurston-book}) states that every geodesic lamination admits a transverse measure (may not be of full measure). The idea behind the proof of this is to approximate a transverse measure using counting measure. A geodesic lamination with a transverse measure is called a measured geodesic lamination. The following theorem by Levitt proves some important properties of leaves of a lamination.

\begin{theorem}[Levitt \cite{levitt}]
  Let $\lambda$ be a geodesic lamination on a closed hyperbolic surface $M$ and $p:\tilde{M}\to M$ be the universal cover of $M$. Let $\g\subset \lambda$ be a leaf of the lamination. Then the following are equivalent:
  \begin{enumerate}
    \item $\g$ is isolated
    \item Each geodesic in $p^{-1}(\g)$ is an isolated point in $G_\lambda = p^{-1}(\lambda)\subset \tilde{M}$.
    \item The complement of $\g$ in $\lambda$ is compact.
  \end{enumerate}
  If $\g$ is non-compact then 1,2,3 are also equivalent to 
  \begin{enumerate}
    \setcounter{enumi}{4}
    \item $\g$ is proper
    \item $\g$ belongs to no minimal sublamination of $\lambda$
    \item No transverse measure of $\lambda$ contains $\g$ in its support.
  \end{enumerate}
  Moreover, there can only be finitely many isolated leaves in a lamination; the other leaves are partitioned into finitely many minimal sublaminations.
\end{theorem}
If $\lambda$ is a measured geodesic lamination of full support then, as a consequence of the above theorem, it cannot have any leaves which are spiraling onto compact leaves as they are isolated and thus cannot be in the support. A non-compact leaf $\g$ which does not spiral onto a compact leaf belongs to the minimal sublamination $\overline{\g}$ (closure of the leaf in $M$).\\

The space of all measured laminations with full support is denoted $ \mathcal{ML}(M)$. Then the following theorem is due to Thurston:
\begin{theorem}
  The space of measured foliations on $M$ is homeomorphic to the space of measured laminations on $M$. 
\end{theorem}
\begin{rem}
  The theorem above claims that $ \mathcal{MF}(M)$ and $ \mathcal{ML}(M)$ are homeomorphic but I have not defined a topology on these spaces. This is because defining the topology on these takes some work and it will not be relevant to the remainder of this thesis. For a more rigorous treatment of this see \cite{thurston-book, FLP}.
\end{rem}
\begin{rem}
  In \cite{thurston-book}, a proof of the above theorem is given using \textit{train tracks}. A more general study of the relationship between geodesic laminations and foliations (both measured and non-measured) can be found in \cite{levitt}. 
\end{rem}

\section{Maps between Hyperbolic Surfaces}
This section is devoted to understanding and highlighting some important properties of homotopy equivalence between hyperbolic surfaces. Throughout this section let $X,Y$ be compact hyperbolic surfaces which are homeomorphic to a topological surface $S$, $p_X:\tilde{X}\to X \And p_Y:\tilde{Y}\to Y$ be their universal covers, and $f:X\to Y$ be a homotopy equivalence. Let $f_{*}:\pi_1(X)\to \pi_1(Y)$ be the induced isomorphism on the fundamental group. Denote by $\Gamma_X$ and $\Gamma_Y$ some holonomy representation of $\pi_1(X)$ and $\pi_1(Y)$ in $PSL_2(\R)$. Whenever $X,Y$ are assumed to be closed their universal covers are identified with $\H$.
\begin{proposition}
  Let $f:X\to Y$ be a homotopy equivalence. Then any lift $\tilde{f}:\tilde{X}\to \tilde{Y}$ is a homotopy equivalence and is $(\pi_1(X),\pi_1(Y))-$equivariant: for each $\a\in \pi_1(X)$ we have $f_{*}(\a)\tilde{f} = \tilde{f}\alpha$.
\end{proposition}
This follows directly from basic results in algebraic topology. When $X,Y$ are closed we want to understand when $\tilde{f}$ can be extended to the boundary $\bH$. The following theorem gives us an answer.
\begin{theorem}\label{thm:boundary_maps}
  Let $f:X\to Y$ be a homotopy equivalence between closed hyperbolic surfaces. Then any lift $\tilde{f}$ of $f$ has a unique equivariant extension $\bar{f}$ on $\H\cup \bH$ such that the restriction of this map to the boundary $\pa f:\bH\to \bH$ is a homeomorphism invariant under homotopies of $f$.
\end{theorem}
\begin{proof}[Proof Idea.]
  By Milnor-Schwarz lemma, the orbit maps of the fundamental group of $X,Y$ at $0\in \H$,   $q_X:\Gamma_X\to \H$ and $q_Y:\Gamma_Y\to \H$, are quasi-isometries with respect to the word metric and $f_*$ is bi-Lipschitz. Due to equivariance of $\tilde{f}$ the diagram
  \[\begin{tikzcd}
	{\Gamma_X} && {\Gamma_Y} \\
	\\
	{\mathbb{H}^2} && {\mathbb{H}^2}
	\arrow["{f_*}", from=1-1, to=1-3]
	\arrow["{q_X}"', from=1-1, to=3-1]
	\arrow["{q_Y}", from=1-3, to=3-3]
	\arrow["{\tilde{f}}"', from=3-1, to=3-3]
\end{tikzcd}\]
commutes. This implies that $\tilde{f}$ is a quasi-isometry on $\H$, and thus it maps geodesics to quasi-geodesics. For any point $x\in \bH$ let $\pa {f}(x)$ be the end-point of the quasi-geodesic $f(\a)$ where $\a$ is the extension of the geodesic $[0,x]$. Since quasi-geodesics are always a bounded distance away from geodesics (using stability of quasi-geodesic theorem in \cite{metric}), this endpoint always exists. The remaining properties can be proven without much difficulty.
\end{proof}
\begin{rem}
  The proof of this theorem in \cite{casson} does not use Milnor-Schwarz lemma and arrives at the same conclusion using elementary analysis and geometry. Although, using the same idea as in the proof presented above this theorem can also be generalized to surfaces with boundary by replacing $\pa \H$ with the Gromov boundary $\pa \tilde{X}$ and $\pa \tilde{Y}$. 
\end{rem}
The following is a direct Corollary of \ref{thm:boundary_maps} for closed surfaces, but I have stated it in a way so that it works for surfaces with boundaries as well. 
\begin{proposition}\label{pro:isometries}
  Let $X$ be a compact hyperbolic surface (with or without boundary) and $I:X\to X$ be some isometry which is isotopic to the identity. Then there is a lift $\tilde{I}:\tilde{X}\to \tilde{X}$ of $I$ which is the identity map.
\end{proposition}
\begin{proof}
  Let $H:X\times I\to X$ be an isotopy such that $H_0 = I$ and $H_1 = \text{id}_X$. Then there is a unique lift of $H$ to an isotopy $\tilde{H}:\tilde{X}\times I \to \tilde{X}$ such that $\tilde{H}_1 = \text{id}_{\tilde{X}}$. Let $\tilde{I} = \tilde{H}_0$. Using the fact that $p:\tilde{X}\to X$ is a local isometry and the fact that $X$ is compact, one can conclude that $\tilde{I}$ is an isometry which is isotopic to $\text{id}_{\tilde{X}}$, and moreover it is a bounded distance away from $\text{id}_{\tilde{X}}$. Let $\text{Dev}:\tilde{X}\to \H$ be some developing map. Then the map $\text{Dev}\circ \tilde{I}$ is also a developing map. It follows from the discussion in \ref{sec:gx} that there is some $g\in \PSL$ such that $\text{Dev}\circ \tilde{I} = g\text{Dev}$. Since $\tilde{I}$ is a bounded distance away from $\text{id}_{\tilde{X}}$ it follows that $g$ is a bounded distance away from $\text{id}_{\H}$. Thus it follows that $g$ and $\text{id}_{\H}$ agree on $\bH$, but since any Mobius map is determined by what it does to exactly three points one concludes that $g=\text{id}_{\H}$. Hence $\tilde{I}$ is just the identity.
\end{proof}
\begin{cor}
  If $I_1, I_2:X\to Y$ are isometries of compact hyperbolic surfaces (with or without boundary) such that $I_1$ is isotopic to $I_2$ then for any lift $\tilde{I}_1$ of $I_1$ to the universal cover there is a lift $\tilde{I}_2$ of $I_2$ such that the two lifts are equal.
\end{cor}
\begin{theorem}\label{thm:laminations}
  Let $f:X\to Y$ be a homotopy equivalence. Then there is a bijective correspondence between geodesic laminations on $X$ and $Y$.
\end{theorem}
\begin{proof}[Sketch of Proof.]
  Suppose that $\lambda$ is a geodesic lamination on $X$ which lifts to a lamination $\tilde{\lambda}$ of $\H$. Each leaf of $\tilde{\lambda}$ meets $\bH$ at two points, say $(a,b)$. Using the boundary map $\pa f$ from \ref{thm:boundary_maps}, we get two points $(f(a),f(b))$ on $\bH$, which determines a unique geodesic in $\H$. This gives us a collection of disjoint geodesics $\tilde{\lambda}'$ on $\bH$. This indeed forms a geodesic lamination, which is then pushed down to a geodesic lamination $\lambda' = p_Y(\tilde{\lambda}')$ on $Y$. 
\end{proof}
\begin{rem}
  In fact, the space of geodesic laminations on $X$ and $Y$ can be proven to be homeomorphic under the topology induced by the Hausdorff metric. The proof of that can be found in \cite{casson}. 
\end{rem}
\begin{definition}
  A map $f:X\to Y$ is said to be \textit{Lipschitz continuous} if there is a $K$ such that $d_Y(f(x),f(x'))\leq Kd_X(x,x')$ for all $x,x'\in X$. A Lipschitz map $f$ is said to be $K-$Lipschitz if
  $$K = \sup_{x\neq x'\in X}\frac{d_Y(f(x),f(x'))}{d_X(x,x')}$$
  $K$ is called the Lipschitz constant of $f$ and is sometimes denoted $\text{Lip}(f)$. $f$ is called bi-Lipschitz if it has an inverse which is also Lipschitz.
\end{definition}
\begin{proposition}
  Every diffeomorphism $f:X\to Y$ is a Lipschitz map with Lipschitz constant $K = \sup_{x\in X}\|\dd f_x\|_{\text{op}}$.
\end{proposition}
\begin{proof}
  Let $\g$ be any piece-wise smooth curve on $X$ connecting the points $x$ and $x'$. Since $\dd f:\text{T}^1X\to \text{T}Y$ is smooth, it follows by compactness of $X$ that $x \mapsto \|\dd f_x\|_{\text{op}}$ has an upper bound, say $K$. Then,
  \begin{align*}
    \text{Length of}\ f(\g) &= \int_0^1 \|\dot{f(\g)}(t)\|_Y \dd t \\ &= \int_0^1 \|df_{\g(t)}(\dot{\g}(t))\|_Y\dd t\\ &\leq K \int_0^1 \|\dot{\g}(t)\|_X \dd t\\ &= K\cdot (\text{Length of}\ \g)
  \end{align*}
  Taking the infimum over all curves $\gamma$, we get that $f$ is Lipschitz. Following the same argument for $f^{-1}$ it can be shown that $f$ is bi-Lipschitz.
\end{proof}
\begin{rem}
  Since every homeomorphism of surfaces is isotopic to a diffeomorphism of those surfaces, it follows that every homeomorphism is isotopic to a bi-Lipschitz map.
\end{rem}
\begin{theorem}[Theorem 8.9 in \cite{primer}]\label{thm:homotopy-eq}
  If $S$ is closed surface with genus $g\geq 2$ then every homotopy equivalence $f:S\to S$ is homotopic to a homeomorphism of $S$. 
\end{theorem}
\begin{rem}
This theorem also applies to compact surfaces with boundaries whenever the homotopy equivalence restricts to a homeomorphism $\pa S\to \pa S$.
\end{rem}
