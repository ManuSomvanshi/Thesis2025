\section{Hyperbolic Plane}
\begin{definition}
  The upper half plane model of the hyperbolic plane is the set of all points in $\C$ with positive imaginary part, $\Im(z)> 0$ with the metric given by
  \begin{align*}
    \dd s^2 = \frac{(\dd x^2 + \dd y^2)}{y^2}
  \end{align*}
  We denote this by $ \mathbb{H}$. 
\end{definition}

\begin{definition}
  The hyperbolic length of a piecewise differentiable curve $\g:[0,1]\to \mathbb{H}$ is given by
  \begin{align*}
    h(\g) = \int_0^1 \frac{|\dot{\g}(t)|}{\Im(\g(t))} \dd t
  \end{align*}
  The distance between two points $z_1, z_2$ is given by,
  \begin{align*}
    \rho(z_1,z_2) = \inf_{\g} h(\gamma),\ \g(0) =z_1 \qq{and} \g(1) = z_2
  \end{align*}
  this is well defined since $\C$ is path connected.
\end{definition}
\begin{proposition}
  $\rho$ is a metric.
\end{proposition}
\begin{proof}
  Suppose that $\rho(z_1,z_2) = \ell_1$ and $\rho(z_2,z_1) = \ell_2$. Now let $\gamma$ be any curve from $z_2$ to $z_1$. Then $\g(1-t)$ is a curve from $z_1$ to $z_2$ and since $h(\g) = h(\g(1-t))\geq \ell_1$ it follows that $\ell_1$ is also a lower bound and we must have $\ell_1 \leq \ell_2$. Repeating the same in the opposite direction we have $\ell_2\leq \ell_1$. Hence $\rho$ is symmetric.\\

  Let $z_1,z_2, z_3$ be three points. Represent curves from $z_1 \to z_2$ by $\g$, $z_2\to z_3$ by $\sigma$ and $z_1\to z_3 by \tau$. Given any $\g$ and $\sigma$ we can construct a curve from $z_1$ to $z_3$ in following way
  \begin{align*}
    \tau(t) = \g * \sigma(t) = \begin{cases}
      \g(2t),\ 0\leq t\leq 1/2\\
      \sigma(2t-1),\ 1/2 \leq t\leq 1.
    \end{cases}
  \end{align*}
  The length $h(\g*\sigma) = h(\g) + h(\sigma)$ (linearity and change of variables). It follows
  \begin{align*}
    \inf_\tau h(\tau) &\leq \inf_{\g, \sigma} h(\g*\sigma) = \inf_\g h(\g) + \inf_\sigma h(\sigma)\\
    \implies \rho(z_1,z_3) &\leq \rho(z_1,z_2) + \rho(z_2,z_3).
  \end{align*}
  Hence $\rho$ satisfies triangle inequality.\\

  Clearly $\rho(z,z) = 0$. Let $z_1, z_2$ be two points and $\g$ be a curve between them. Then
  \begin{align}
    h(\g) &= \int_0^1 \f{|\dot{\g}|}{\Im(\g)} \dd t \geq \abs{\int^1_0 \f{\dot{\g}}{\Im(\g)} \dd t} \nonumber\\
          &> \f{1}{M}\abs{\int^1_0 \dot{\g}\dd t} = \f{1}{M} |z_2-z_1|
  \end{align}
  Where $M = \sup_{t\in [0,1]}(Im(\g(t)))$ (this is well defined since its a continuous function on a compact interval). This completes the proof.
  Suppose $\rho(z_1,z_2) = 0$ then $h(\g)< 0$ and hence $|z_1 - z_2| < 0$. Hence $z_1 = z_2$.
\end{proof}
\begin{proposition}
  The set of all Mobius transforms from $\C\to \C$ of the form
  \begin{align*}
    z \mapsto \f{az+b}{cz+d},\ a,b,c,d\in \R\ ad-bc =1
  \end{align*}
  form a group, under composition. This group is isomorphic to $PSL_2(\R) = SL_2(\R)/\{\pm I\}$.
\end{proposition}
\begin{proof}
  The first part is trivial. For the second consider the map from $SL_2(\R)$
  \begin{align*}
    \begin{pmatrix}
      a&b\\c&d
    \end{pmatrix} \mapsto \l(z\mapsto \f{az+b}{cz+d}\r)
  \end{align*}
  This is clearly surjective. Suppose that
  \begin{align*}
    \f{az+b}{cz+d} = z \implies -cz^2 + (a-d)z + b = 0 \implies c = 0,\ a=d,\ b=0.
  \end{align*}
  Since $ad-bc=1$ it follows that $a = \pm 1$. Hence $\ker$ of the map is just $\pm I$.
\end{proof}
Note that any Mobius transformation in $PSL_2(\R)$ can be written by composing the functions $z\mapsto az$, $z\mapsto z+b$, $z\mapsto -1/z$.
\begin{proposition}
  The metric topology of $\H$ is equivalent to the subspace topology induced from $\C^2$.
\end{proposition}
\begin{proof}
  Consider the two points $z,w\in\H$. Consider the curve $\g(t) = (z-w)t + w$. By definition
  \begin{align*}
    \rho(z,w) \leq \int^1_0 \f{|\dot{\g}|}{\Im(\g)} \dd t = |z-w| \int^1_0 \f{\dd t}{(y-v)t+v} \leq |z-w| \int^1_0 \f{\dd t}{\min\{y,v\}} = \f{|z-w|}{\min\{y,v\}}  
  \end{align*}
  Where $\Im(z) = y$ and $\Im(w) = v$. Hence $\rho(z,w) \leq K_w |z-w|$ for some $K_w>0$. From (1) we already know that
  \begin{align*}
    |z-w| \leq M \rho(z,w) 
  \end{align*}
  Hence the metrics are equivalent. 
\end{proof}
This is an important result since we can just check the continuity of functions under the regular metric. This helps reduce calculations.
\begin{theorem}
  $PSL_2(\R)$ acts on $ \mathbb{H}$ by homeomorphisms.
\end{theorem}
\begin{proof}
  Suppose $z\in\H$ and
  \begin{align*}
    w = \f{az+b}{cz+d}
  \end{align*}
  then
  \begin{align*}
    w-\bar{w} &= \f{az+b}{cz+d} - \f{a\bar{z}+b}{c\bar{z}+d}\\
              &= \f{z-\bar{z}}{|cz+d|^2} > 0
  \end{align*}
  Thus points on $\H$ are mapped to $\H$ itself. Since mobius transformations are bijective, continuous, and the inverse again is a mobius transformation it follows that it is a homeomorphism on $\H$. 
\end{proof}
\begin{theorem}
  $PSL_2(\R)$ is isomorphic to a subgroup of the isometry group of $\H$.
\end{theorem}
\begin{proof}
  Let $z,w\in \H$ and $T\in PSL_2(\R)$ be a mobius transformation $(a,b,c,d)$. Let $\g$ be a curve from $z$ to $w$ then $\sigma(t) = T(\g(t))$ is a curve from $Tz$ to $Tw$. Since
  \begin{align*}
    \dot{\sigma}(t) = \f{\dot{\g}}{(c\g + d)^2},
  \end{align*}
  and
  \begin{align*}
    \Im{\sigma(t)} = \f{\Im{\g}}{|c\g(t)+d|^2}
  \end{align*}
  it follows that
  \begin{align*}
    h(\sigma) = h(\g).
  \end{align*}
  Since $T$ is bijective there is a bijective correspondence between curves from $z$ to $w$ and curves from $Tz$ to $Tw$. Thus they have the same infimum, meaning that $\rho(w,z) = \rho(Tw,T,z)$. 
\end{proof}
\begin{definition}
  Geodesics are curves with the shortest length between any two points in a metric space.
\end{definition}
\begin{theorem}
  The geodesics in $\H$ are straight lines and semi circles perpendicular to the real axis. Moreover, between any two points in $\H$ there exists a unique geodesic.
\end{theorem}
\begin{proof}
  Consider the two points $z = a_0 + ia$ and $w = a_0 + ib$ in $\H$. For any curve $\g(t) = x(t) + iy(t)$ between these points,
  \begin{align}
    h(\g) &= \int^1_0 \f{\sqrt{\dot{x}^2 + \dot{y}^2}}{y(t)} \dd t \nn\\
          &\geq \int^1_0 \f{|\dot{y}|}{y(t)}\dd t \nn\\
          &\geq \abs{\int^1_0 \f{\dot{y}}{y} \dd t} \nn \\
          &= \abs{\log(\f{b}{a})}
  \end{align}
  But since the curve $\g_0(t) = i(b-a)t + ia + a_0$ also has the length $|\log(b/a)|$ it follows that $\rho(z,w) = |\log(b/a)|$. The geodesic, $\g_0$, in this case is a straight line perpendicular to $\R$.\\

  Now consider any two points $z_1,z_2\in \H$. Then there is a unique circle which passes through $z_1,z_2$ and is perpendicular to the real line: $|z-a| = r$ where
  \begin{align*}
    a = \f{|z_1|^2 - |z_2|^2}{2(\Re(z_1-z_2))} \And r = |z_1 - a|. 
  \end{align*}
  There also exists a $T\in PSL_2(\R)$ which maps the above semi-circle in $\H$ to the positive imaginary line. Explicitly this is:
  \begin{align*}
    T = \f{1}{\sqrt{2r}}\begin{pmatrix}
      1 & -a-r \\ 1 & -a +r 
    \end{pmatrix}
  \end{align*}
  Suppose that $z$ lies on the semi-circle, then $z-a = re^{i\theta}$. Thus under the transformation
  \begin{align*}
    T(z) = \f{e^{i\theta}-1}{e^{i\theta}+1} \in i\R^+
  \end{align*}
  Hence the semi circle gets mapped to the imaginary axis, which is a geodesic. Since $T$ is an isometry it follows that the semicircle is also a geodesic. The uniqueness follows from the fact that for any curve other than the straight line the inequality in $2$ is strict, and hence only the straight line achives the minimum length. By isometry argument it generalizes to the aribtrary case.
\end{proof}
\begin{definition}
  The set of points on the unique geodesic connecting $w$ and $z$ is represented by $[w,z]$.
\end{definition}
\begin{corollary}
  Let $z,w,\xi\in \H$ then,
  \begin{align*}
    \rho(z,w) = \rho(z,\xi) + \rho(\xi,w)
  \end{align*}
  if and only if $\xi\in [z,w]$.
\end{corollary}
\begin{proof}
  Suppose that $\xi\in [z,w]$. Then the geodesic $\g$ from $z$ to $w$ on restriction gives a geodesic between $z$ and $\xi$, because if not then there is some other $\sigma$ between $z,\xi$ such that $h(\sigma) < h(\g|_{[z,\xi]})$, and then the curve
  \begin{align*}
    \tilde{\g}(t) = \begin{cases}
      \sigma(2t),\ 0\leq t\leq 1/2\\
      \gamma\big|_{[\xi,w]} (2t-1),\ 1/2\leq t\leq 1  
    \end{cases}
  \end{align*}
  has smaller length, in contradiction to the fact that $\g$ is the geodesic. Hence
  \begin{align*}
    \rho(z,w) = h(\g) = h(\g\big|_{[z,\xi]}) + h(\g\big|_{[\xi,w]}) = \rho(z,\xi) + \rho(\xi,w).
  \end{align*}
  Conversly, suppose that the equality holds. Let $\g_1,\g_2$ be the geodesics between $z,\xi$ and $\xi,w$ respectively. The concatenation as above defines a curve between $z$ and $w$, and
  \begin{align*}
    \rho(z,w) = \rho(z,\xi) + \rho(\xi,w) = h(\g_1) + h(\g_2) = h(\g_1*\g_2)
  \end{align*}
  Hence the concatenation is a geodesic. Thus $\xi\in [z,w]$.
\end{proof}
\begin{theorem}
  $PSL_2(\R)$ maps geodesics to geodesics.
\end{theorem}
\begin{proof}
  Since, as seen already in theorem 2.7, $h(T\g) = h(\g)$ for all $T\in PSL_2(\R)$ and curves $\g$ between $z,w$. Thus it follows trivially that if $\g$ is a geodesic then so is $T\g$ 
\end{proof}
\begin{definition}
  A cross ratio, denoted $(z_1,z_2;,z_3,z_4)$ is defined as
  \begin{align*}
    (z_1,z_2;z_3,z_4) = \f{(z_1 - z_2)(z_3-z_4)}{(z_1-z_4)(z_3-z_2)}
  \end{align*}
\end{definition}
\begin{proposition}
  Cross ratios are preserved under Mobius transformations.
\end{proposition}
\begin{proof}
  The transformation $T$, given by
  \begin{align*}
    T(z) = \f{z-z_2}{z-z_4} \f{z_3-z_4}{z_3 -z_2},
  \end{align*}
  maps $z_2\mapsto 0$, $z_4\mapsto \infty$, $z_3\mapsto 1$. $T(z_1)$ is exactly the cross ratio. Let $S$ be any mobius transformation then $TS^{-1}$ is the transformation which maps $Sz_2 \mapsto 0$, $Sz_4\mapsto \infty$, and $Sz_3\mapsto 1$. Hence $TS^{-1}(z) = (z,Sz_2; Sz_3, Sz_4)$. Hence $(Sz_1,Sz_2;Sz_3,Sz_4) = TS^{-1}(Sz) = T(z) = (z_1,z_2;z_3,z_4)$.
\end{proof}
\begin{theorem}
  Let $w,z\in \H$ and $\g$ be the geodesic between them. Extend $\g$ in both directions and let $w^*, z^*\in \R\cup \{\infty\}$ be the end points of the extended curve (semi-circle or straight line perpendicular to real axis) such that $z\in [z^*,w]$. Then
  \begin{align*}
    \rho(w,z) = \log(w,z^*; z,w^*)
  \end{align*}
\end{theorem}
\begin{proof}
  As seen before there exists a $T\in PSL_2(\R)$ such that $T$ maps the extended $\g$ to the imaginary axis. Explicitly such a $T$ is
  \begin{align*}
    T(\xi) = i\f{\xi-z^*}{\xi-w^*}\cdot \f{z-w^*}{z-z^*} 
  \end{align*}
  this maps $z^*$ to $0$, $w^*$ to $\infty$, and $z$ to $i$. Note that the coefficient of $T$ are indeed all real, and it can be made determinant $1$ by multiplying and dividing by a real constant. And moreover,
  \begin{align*}
    T(w) = i\underbrace{\f{w-z^*}{w-w^*}\cdot \f{z-w^*}{z-z^*}}_{r}.
  \end{align*}
  $r$ must be greater than $1$ since $|z-w^*|>|w-w^*|$ and $|w-z^*|>|z-z^*|$ (since we choose $w$ to be closer to $w^*$ and $z^*$ is closer to $z$). The hyperbolic distance between $T(z)$ and $T(w)$ is $\log(r)$. Since $r = (ir,0;i,\infty) = (T(w), T(z^*); T(z), T(w^*)) = (w,z^*;z,w^*)$. Hence the statement of the theorem follows. 
\end{proof}

Now we describe the Poincare Disk model of the Hyperbolic plane. Consider the unit disk $\mathbb{D}$, and the map $\p:\H\to \mathbb{D}$
\begin{align*}
  \phi(z) = \f{iz+1}{z+i}
\end{align*}
Clearly $|\p(z)| = |z-i|/|z+i| < 1$ if and only if $z\in \H$. Also $\p$ maps the real line to the boundary of $ \mathbb{D}$. This map induces a distance $\rho^*$ on $ \mathbb{D}$ given by
\begin{align*}
  \rho^*(w,z) = \rho(\p^{-1}(w), \p^{-1}(z))
\end{align*}
It follows that,
\begin{align*}
  \rho^*(w,z) &= \inf_{\g} \int^1_0 \f{|\dv{\p^{-1}\circ \g}{t}|}{\Im(\p^{-1}\circ \g)} \dd t\\
          &= \inf_{\g} \int^1_0 \f{|\dv{\p^{-1}}{z}\big|_\g \dot{\g}|}{\Im(\p^{-1}\circ \g)} \dd t\\
          &= \inf_\g \int^1_0 \f{2|\dot{\g}(t)|}{1- |\g(t)|^2} \dd t
\end{align*}
This gives the metric on $ \mathbb{D}$ to be
\begin{align*}
  \dd s = \f{2|\dd z|}{1-|z|^2}.
\end{align*}
This model of the hyperbolic plane is called the Poincare Disk. The geodesics here are circles perpendicular to $ \mathbb{D}$ and diametric lines in $ \mathbb{D}$.
\begin{definition}
  Let the group of all $2\times 2$ matrices in $\R$ with determinant $\pm 1$ be denoted $S^*L_2(\R)$. Let $PS^*L_2(\R)$ be the group $S^*L_2(\R)/\{\pm I\}$.
\end{definition}
\begin{proposition}
  Let $z,w\in \H$. Then
  \begin{align*}
    \sinh(\f{1}{2} \rho(z,w)) = \f{|z-w|}{2\sqrt{\Im(z)\Im(w)}}
  \end{align*}
\end{proposition}
\begin{proof}
  Let $T$ be in $PSL_2(\R)$ then $T$ leaves the LHS invariant. It is straight forward to check that the RHS is also invariant. Suppose $z = ia$ and $w = ib$ ($b>a$) then we know that $\rho(ia,ib) = \log(b/a)$ and thus
  \begin{align*}
    \sinh(\f{1}{2}\rho(ia,ib)) = \f{b-a}{2\sqrt{ab}} = \f{|ib| - |ia|}{2\sqrt{\Im(ia)\Im(ib)}}
  \end{align*}
  Using the fact that there exists a $T$ which maps the geodesic between arbitrary $z,w$ to a geodesic between $ia,ib$; the result follows.
\end{proof}
\begin{theorem}
  The isometry group of $\H$ is isomorphic to $PS^*L_2(\R)$.
\end{theorem}
\begin{proof}
  Let $\p$ be any isometry. If $\xi\in [z,w]$ then
  \begin{align*}
    \rho(\p(z), \p(z)) = \rho(z, w) = \rho(z,\xi) + \rho(\xi,w) = \rho(\p(z), \p(\xi)) + \rho(\p(\xi), \p(w))
  \end{align*}
  Thus $\p(\xi) \in [\p(z), \p(w)]$. This means that isometries map geodesics to geodesics. Consider the positive imaginary line $I$ which is a geodesic. Then $\p(I)$ is also some geodesic. There exists a $g\in PSL_2(\R)$ such that $g$ maps $\p(I)$ to $I$. Without loss of generality we can assume $g(\p(i)) = i$ (since $g(\p(i)) = ai$, and dividing by $a$ we get another element of $PSL_2(\R)$) and that it maps $(0,i)$ and $(i,\infty)$ onto themselves (like in the previous theorem). Suppose that $g\circ\p(yi) = vi$ then
  \begin{align*}
    |\log(y)| = \rho(yi, i) = \rho(g\circ\p(yi), i) = |\log(v)|
  \end{align*}
  Either $v = y$ or $v = 1/y$, but since the intervals $(0,i)$ and $(i,\infty)$ are fixed it follows that $v = y$. Hence $g\circ\p$ fixes $I$. Let $\g\circ\p(x+iy) = u+iv$ then using the previous proposition on the points $x+iy$ and $it$ we get
  \begin{align*}
    \f{x^2 + (y-t)^2}{2y} = \f{u^2 + (v-t)^2}{2v} 
  \end{align*}
  dividing by $t^2$ and taking $t\to \infty$ we get $v = y$ and $x^2 = u^2$. Thus
  \begin{align*}
    g\circ\p(z) = z \qq{or,} -\bar{z}
  \end{align*}
  In the first case $\p$ is in $PSL_2(\R)$ and in the second case it is of the form
  \begin{align*}
    \p(z) = g^{-1}(-\bar{z}) = \f{a\bar{z} + b}{c\bar{z} + d},\ ad-bc = -1
  \end{align*}
  Thus $\p$ can be naturally mapped to an element of $S^*L_2(\R)$. The homomorphism part of the mapping follows easily, and the kernel of the map is $\{\pm I\}$. Thus the isometry group is isomorphic to $PS^*L_2(\R)$.
\end{proof}
Note that $PSL_2(\R)$ along with the map $h:z\mapsto -\bar{z}$ generates the isomoetry group. This means that the quotient space $PS^*L_2(\R)/PSL_2(\R)$ is just $\{PSL_2(\R), h\cdot PSL_2(\R)\}$ and thus has index $2$. This means that $PSL_2(\R)$ is normal in the isometry group.\\

The Riemannian metric of the Hyperbolic plane is induced by the inner product $\langle \cdot, \cdot \rangle: T_z\H \times T_z\H \to \R$ given by
\begin{align*}
  \langle \zeta_1, \zeta_2 \rangle = \f{1}{\Im(z)^2} \Re(\zeta_1 \bar{\zeta}_2)
\end{align*}
This is an inner product on $T_z\H$ over $\R$. This induces a norm $\|\cdot\|$ on $T_z\H$ defined as
\begin{align*}
  \|\zeta\| = \sqrt{\langle \zeta, \zeta \rangle} = \f{|\zeta|}{\Im(z)}
\end{align*}
Since all isometries of $\H$ are (real) differentiable, their pushforward gives a the map $\dd \p_z : T_z\H \to T_{\p(z)}\H$ 
\begin{align*}
  \dd\p_z(\zeta) = \f{\pm \zeta}{(cz+d)^2},\ \text{where},\ \p(z) = \f{az+b}{cz+d}, \And ad-bc = \pm 1. 
\end{align*}
The pushforward is norm preserving since
\begin{align*}
  \|\dd \p_z(\zeta)\| = \f{|\zeta|}{|cz+d|^2 \Im(\p(z))} = \f{|\zeta|}{\Im(z)} = \|\zeta\|
\end{align*}
Using the polarization identity,
\begin{align*}
  \langle \zeta_1, \zeta_2 \rangle = \f{1}{2}(\|\zeta_1\| + \|\zeta_2\| - \|\zeta_1 - \zeta_2\|)
\end{align*}
we can conclude that the pushforward of isometries preserve the absolute value of angles between vectors.
\begin{definition}
  Angle between geodesics in $\H$ is defined as the angle between the tangent vectors at the point of intersection.
\end{definition}
\begin{definition}
  A map on $\H$ is said to be confromal if it preserves angles, and anti-conformal if it preserves absolute value of the angle but reverses direction.
\end{definition}
\begin{theorem}
  Transformations in $PSL_2(\R)$ are conformal and the other isometries are anti-conformal.
\end{theorem}
\begin{proof}
  We saw already that the pushforward preserves the absolute value of angles. But since the pushforward at each point is of the form
  \begin{align*} 
  \dd\p_z(\zeta) = \f{\pm \zeta}{(cz+d)^2},\ \text{where},\ \p(z) = \f{az+b}{cz+d}, \And ad-bc = \pm 1. 
  \end{align*}
  it follows that $PSL_2(\R)$ preserves direction while $z\mapsto -\bar{z}$ reverses orientation.
\end{proof}
\begin{definition}
  Hyberbolic area of a subset $A$ of $\H$ is defined as
  \begin{align*}
    \mu(A) = \int_A \f{\dd x \dd y}{y^2}
  \end{align*}
\end{definition}
\begin{theorem}
  If $A\subset \H$ and $\mu(A)$ exists then the hyperbolic area is invariant under transformations of $PSL_2(\R)$.
\end{theorem}
\begin{proof}
  Suppose that $z = x+iy$ and $Tz = u+iv$. Then using the Cauchy-Riemann equations the determinant of the Jacobian $\pa (u,v)/\pa (x,y)$ is given by
  \begin{align*}
    \abs{\f{\pa (u,v)}{\pa (x,y)}} &= \pdv{u}{x}\pdv{v}{y} - \pdv{u}{y}\pdv{v}{x}\\
                                   &= \l(\pdv{u}{x}\r)^2 + \l(\pdv{v}{x}\r)^2\\
                                   &= \abs{\dv{T}{z}}^2\\
                                   &= \f{1}{|cz+d|^4}
  \end{align*}
  Thus by change of variables
  \begin{align*}
    \mu(T(A)) = \int_{T(A)} \f{\dd u \dd v}{v^2} = \int_{A}  \f{|cz+d|^4}{|cz+d|^4 y^2} \dd x \dd y  = \mu(A)
  \end{align*}
  Thus the area is invariant under $PSL_2(\R)$.
\end{proof}
\begin{definition}
  An $n-$sided polygon in $\H\cup\R\cup\{\infty\}$ is defined by the area enclosed by $n$ distinct geodesics. The vertices can lie on the boundary.
\end{definition}
\begin{theorem}[Gauss-Bonnet Theorem]
  A hyperbolic triangle $\Delta$ with angles $\a, \b, \g$ has area $\mu(\Delta) = \pi - \a - \b -\g$.
\end{theorem}
\begin{proof}
  \textit{Case 1.} Consider a triangle with one point on $\R\cup\{\infty\}$ then there exists a transformation in $PSL_2(\R)$ which takes the vertex on $\R\cup \{\infty\}$ to $\infty$. Thus w.l.o.g. we consider a triangle with two sides being lines perpendicular to the imaginary axis. Again w.l.o.g. (by a $PSL_2(\R)$ transformation) we can translate and scale the triangle such that the center of the semi-circle (the third side) is at $0$ with radius $1$. All these transformations preserve the area and the angles. 
\begin{figure}[h]
  \centering
  \includegraphics[scale=0.5]{figures/gauss-bonnet.png}
  \caption{Case 1.}
\end{figure}
The area of this triangle can be calculated
\begin{align*}
  \mu(\Delta) &= \int_{\Delta} \f{\dd x \dd y}{y^2}\\
       &= \int^b_a \dd x \int_{\sqrt{1-x^2}}^\infty \f{\dd y}{y^2}\\
       &= \int^b_a \f{\dd x}{\sqrt{1-x^2}}\\
       &= \int_{\pi-\a}^\b \f{-\sin\theta \dd \theta}{\sin\theta}\\
       &= \pi - \a - \b.\\
\end{align*}
\textit{Case 2.} Suppose that none of the vertices are on $\R\cup \{\infty\}$. There exists a transforamtion such that no two vertices lies on a vertical geodesic. Extend the side $AB$, of the triangle $ABC$, to a point $D\in \R$. Let $\Delta_1 = ACD$ and $\Delta_2 = CBD$. Then
\begin{align*}
  \mu(\Delta) = \mu(\Delta_1) - \mu(\Delta_2) = \pi - \a - (\g+\theta) - \pi + \theta + (\pi - \b) = \pi - \a - \b - \g
\end{align*}
\begin{figure}[h]
  \centering
  \includegraphics[scale=0.5]{figures/gauss-bonnet-2.png}
  \caption{Case 2.}
\end{figure}
This proves the theorem.
\end{proof}
\begin{corollary}
  The area of an $n-$gon with angles $\a_1\cdots, \a_n$ is $(n-2)\pi - \a_1 - \cdots - \a_n$.
\end{corollary}
\begin{proof}
  This follows from induction. It is true for a trianlge. Suppose it is true for an $n-1$-gon. An $n-$gon can be divided into a triangle and an $n-1$-gon by drawing an appropriate geodesic curve. Adding the areas of the two we get the result.    
\end{proof}
\begin{theorem}[Brouwer's Fixed Point Theorem]
  Let $f: \mathbb{D} \to \mathbb{D}$ be a continous bijection on the disk. There are is atleast one fixed point.
\end{theorem}
\begin{proof}[Amazing proof using Functors]
  Suppose $f$ fixes no point. Define the function $r: \mathbb{D} \to S^1$ in the following way: draw extend the line from $f(x)$ to $x$ to the boundary of $ \mathbb{D}$ where it intersects with the circle at $r(x)$. Note that $r$ fixes each point on $S^1$. We have the short exact sequence
  \[
    \begin{tikzcd}
      0 \arrow[r] & S^1 \arrow[r, "i"]  & \mathbb{D} \arrow[r, "r"]  & S^1 \arrow[r] & 0
    \end{tikzcd}
  \]
  Since there is a functor $ \pi_1: \text{Top}_* \to \text{Grp}$ which maps a based topological space $(X,x_0)$ to it's fundametal group at $x_0$. Applying this functor to the above exact sequence at any point, gives the short exact sequence
  \[
    \begin{tikzcd}
      0 \arrow[r] & \pi_1(S^1) \arrow[r, "\pi_1(i)"]  & \pi_1(\mathbb{D}) \arrow[r, "\pi_1(r)"]  & \pi_1(S^1) \arrow[r] & 0
    \end{tikzcd}
  \]
  Since $\pi_1(S^1) = \Z$ and $\pi_1( \mathbb{D}) = 0$ we get that $\text{id}_{S^1} = \pi_1(r\circ i) = \pi_1(r)\circ\pi_1(i) = 0$, a contradiction. 
\end{proof}
Brouwer's theorem tells us that isometries of $\H$ must fix atleast one point, since $\H$ is isometric to the disk. The following is a classification of
\begin{theorem}
  Let $T$ be an orientation preserving isometry of $\H$. Then one of the following happens:
  \begin{enumerate}
    \item $T$ fixes only one point in $\H$.
    \item $T$ fixes only one point on the boundary of $\H$.
    \item $T$ fixes two points on the boundary of $\H$.
  \end{enumerate}
\end{theorem}
\begin{proof}
  Since $T\in PSL_2(\R)$, $z$ is a fixed point if
  \begin{align*}
    z = \f{az+b}{cz+d} \implies cz^2 + (d-a)z - b = 0
  \end{align*}
  Thus
  \begin{align*}
    z = \f{(a-d) \pm \sqrt{(a+d)^2 -4}}{2c}
  \end{align*}
  Thus case $1$ corresponds to $a+d<2$, $2$ corresponds to $a+d = 2$, and $3$ corresponds to $a+d>2$. 
\end{proof}
