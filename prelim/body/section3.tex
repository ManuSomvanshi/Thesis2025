\section{Hyperbolic Structures on Surfaces}
For the sake of consistency with the texts, I use the Poincare model of hyperbolic plane here.
\begin{definition}
  Let $X$ be a surface. A hyperbolic structure on $X$ is an atlas of charts such that each chart $\p:U\to \p(U)\subset \H$ is a homeomorphism and the transition maps in the atlas are restrictions of orientation preserving isometries of $\H$.
\end{definition}

Let $p\in X$ and $(U,\p)$ be a chart around $p$. We can define an inner product on $T_p X$ as
\begin{align*}
  \langle u,v \rangle_{T_pX} = \langle \p_* u, \p_* v \rangle  
\end{align*}
This is well defined since if $(V,\psi)$ is another chart containing $p$ then
\begin{align*}
  \langle \psi_* u, \psi_* v \rangle = \langle \psi_*\circ \p_*^{-1}\circ \p_* u, \psi_*\circ\p_*\circ \p_* v \rangle = \langle \p_* u, \p_* v \rangle.
\end{align*}
Since $\langle \cdot, \cdot \rangle_{T_pX}$ is a smooth bilinear map it serves as a Riemann metric on $X$. From here on I drop the subscript $T_pX$ on the metric. The norm induced by the metric will just be written $\|\cdot\|$. From here on we can think of Hyperbolic surface to be a topological surface with a hyperbolic Atlas and a Riemannian metric which is isometric to the hyperbolic metric locally. The metric induces a distance on $X$ which is given by the infimum of the length of the curves.
\begin{proposition}
  Let $\g:[0,1] \to U\subset X$ be a curve in $X$ between $x_0$ and $x_1$. If $\gamma$ is a geodesic then there exist charts $(U_\a,\p_\a)$ such that they cover $\g$ and $\p_\a\circ \g\big|_{U_\a}$ is a segment of a geodesic in $\H$. 
\end{proposition}
\begin{proof}
  Suppose that $\g$ is a geodesic. Let $x$ be a point on $\g$. There exists a neighborhood $U$ of this point which is isometric to $\H$ by the $\p$. Let $\sigma$ be some curve between the end-points of the curve $\p\circ \g$ (which always exist since $\H$ is complete). Then,
  \begin{align*}
    L(\sigma) = L(\p^{-1}\circ \sigma) \geq L(\g) = L(\p\circ\g)
  \end{align*}
  since $\g$ is a geodesic. Thus it follows that $\p\circ \g$ is a geodesic.
\end{proof}

\begin{theorem}[Half of Hopf-Rinow theorem]
  In a complete hyperbolic surface, all geodesics can be extended indefnitely.
\end{theorem}
\begin{proof}
  Suppose that $\g: (-\epsilon, \epsilon)\to X$ is a bounded geodesic in $X$. Then consider a sequence of points $\g(t_n)$ where $t_n \to \e$. Since $X$ is complete the Cauchy sequence $\g(t_n)$ converges to a unique point, say $x_1$. Let $U,\p$ be some chart centered around $x_1$. The image $\p\circ \g$ is a geodesic by previous proposition. Extend this in $\H$ indefnitely. Then the pull back of this extension extends $\g$ at $x_1$ till a new end point $x_2$. Repeating this process one can indefnitely extend $\g$.
\end{proof}
\begin{theorem}
  Any complete, connected, simply-connected hyperbolic surface is isometric to $\H$.
\end{theorem}
\begin{proof}
  Suppose $X$ is a space with the mentioned properties. Consider the maps $E:\H\to X,D:X\to \H$ defined as follows:
  \begin{itemize}
    \item \textit{The exponential.} Choose a point $a\in X$ and a chart $(U, \p)$ such that $\p(a) = 0$. For $x\in \H$ let $\g$ be the geodesic between $0$ and $x$ and then extend the geodesic $\p^{-1}(\g)$. Define $E(x)$ as the point on the extended geodesic such that $\text{dist}(a,E(x)) = \rho(0,x)$.
    \item \textit{The developing.} Fix a point $a\in X$ and a chart $(U,\p)$ around it. There exists a map $D:X\to \H$ such that $D$ is a local isometry and $D\big|_U = \p$. This claim is proven below:
      \begin{proof}[proof of existence of $D$]
        Choose a path $\g$ between points $a,b\in X$. The path can be cover by finitely many convex coordinate charts (due to compactness), say $(U_i,\p_i)$ with $(U_0, \p_0) = (U,\p)$. Refine the covering such that it is minimal (so that $U_i$ only intersects with $U_{i\pm 1}$ and no $U_i$ is contained in $U_j$). Choose points $x_0=a,\cdots,x_i,\cdots,x_n=b$ on $\g$ such that $[x_i,x_{i+1}]\subset U_i$. If the maps $\p_i$ and $\p_{i+1}$ do not agree on $U_i\cap U_{i+1}$, which contains $x_{i+1}$,then there exists an isometry $g$ (unique extension of $\p_{i+1}\circ\p_{i}^{-1}$) such that $g\circ\p_i = \p_{i+1}$ on their intersection. Thus without loss of generality we can assume that all the charts agree on the intersection (by replacing $\p_{i+1}$ with $g\circ \p_1$). Now define $D(b) = \p_n(b)$.\\

        \textit{(Well defined-ness).} Clearly $D$ is not dependent on the choice $x_i$. Now suppose $(U'_i,\p'_i)_{i=0}^m$ is a different set of charts which minimally cover $\g$ with $U_0,\p_0 = U,\p$ and such that the coordinate charts agree on the intersection. We show by induction that whenever $U_i\cap U'_j\neq \emptyset$ then $\p'_j = \p_i$ in the intersection. By construction $U_0 = U_0'$ and $\p'_0 = \p_0$. Since $U'_0\cap U_1 = U_0\cap U_1$ it follows that in this intersection $\p'_0 = \p_0 = \p_1$. This is the base case of the induction. Suppose now that for all $s<j$  if $U'_s\cap U_i \neq \es$ then $\p'_s = \p_i$ in the intersection for all $i$. Consider $U'_j$ and suppose that it intersects with some $U_i$. There are two cases:
        \begin{enumerate}
          \item $U'_{j-1}\cap U_i\neq \es$. In this case consider the intersection $U_i\cap U'_j\cap U'_{j-1}$. In this region $\p'_j= \p'_{j-1} = \p_i$. Restricted to $\p'_j (U_i \cap U'_j)$ the map $g = \p_i\circ (\p'_j)^{-1}$ is in $PSL_2(\R)$. Since on $\p'_j (U_i \cap U'_j \cap U_i)$ the map $g$ is identity it follows that $g$ is identity everywhere in $\p'_j(U_i\cap U'_j)$ (since Mobius maps are fixed by $3$ points). 
          \item $U'_{j-1}\cap U_i = \es$. Then $U'_{j-1}$ intersects $U_{i-1}$ (by construction). In the region $U_{i-1} \cap U_j\cap U_{j-1}$ we have $\p'_j = \p'_{j-1} = \p_{i-1}$ and thus $\p'\circ \p^{-1}_{i-1}$ is identity on infinitely many points. Thus they are the same on $U'_j\cap U_{i-1}$. In the region $U'_j\cap U_i \cap U_{i-1}$ we have $\p'_j = \p_{i-1} = \p_i$. Using the same argument as before we have that $\p_i = \p'_j$ everywhere on $U_i \cap U'_j$.
        \end{enumerate}
        Hence the induction step is complete. Since $b\in U'_m\cap U_n$ it follows that $\p'_m(b) = \p_n(b)$. Hence $D$ is not dependent on the covering of $\g$. Now we need to show that $D$ does not depend on $\g$. If $\g'$ is some other curve. Since $X$ is simply connected it follows that there is a Homotopy $H$ between $\g$ and $\g'$. Using continuity of $H$ there exists an $\e$ so that the curves $H(s,t)$ and $H(s,t+\e)$ can be covered by the same charts. Hence $\p_n(b)$ is the same for both. Thus it follows that $D$ is well defined.
      \end{proof}
  \end{itemize}
  Now that we have these two functions, note that $D\circ E = 1_{\H}$: let $x\in \H$ then $E(x)$ lies on a geodesic $\g$ from $a$ to $E(x)$ such that $\p\circ \g$ is part of the geodesic connecting $0$ and $x$. Let $U_i$ be any minial cover of the geodesic from $a$ to $E(x)$. Then $\p_n(E(x))$ lies on the extension of the geodesic $\p\circ\g$ and $\rho(0,D(E(x))) = \rho(0,x)$ since $D$ and $E$ are local isometries, but there is only one such point on the geodesic: $x$. Hence $D\circ E(x) = x$.\\

  Note that on the image of $E$ in $X$ the map $E\circ D$ is identity. $E(\H)$ is closed and open (since $E$ is local injection it follows by Invariance of domain theorem). Since $X$ is connected the only non-trivial clopen subset is $X$ itself. Thus $E(\H) = X$. Hence $E\circ D = 1_X$.
\end{proof}

As a consequence of the above theorem it follows that the universal cover of any closed hyperbolic surface $X$ is isometric to $\H$. Thus for any hyperbolic surface $X$, we can write it as $\H/\Gamma$ where $\Gamma$ is the fundamental group of the space $X$ (since deck transformations is isomorphic to fundamental group for universal covers). Note that since $\Gamma$ acts on $\H$ by automorphisms, it is a subgroup of $PSL_2(\R)$. Now suppose that $x\in X$, then the pre-image of $x$ under the covering map is a discrete set, which is equivalent to saying that that the orbit of each point is discrete which in turn is equivalent to saying that $\Gamma$ is discrete. This means $\Gamma$ is a Fuchsian group which acts properly discontinuously on $\H$ (deck transformations act properly discontinuously). As a result $\Gamma$ has no elliptic elements.\\

If $\H/\Gamma$ is compact then the Dirichlet domain $D$ of $\Gamma$ in $\H$ is compact, let $\e$ be the maximum distance between the points in the dirichlet domain. Then $\rho(z,\g z) \geq \e$. But for parabolic elements, which are essentially translations, say $z\mapsto z+1$, the distance between $z, z+1$ is smaller than $\e$ for large enough $\Im(z)$. Thus $\Gamma$ can only have hyperbolic elements.

\begin{proposition}
  The torus is not a hyperbolic surface.
\end{proposition}
\begin{proof}
  Suppose it is, then it has $\H$ as it's universal cover. But that would mean that the fundamental group $\Z\oplus \Z$ is a Fuchsian group. But since all abelian Fuchsian groups are cyclic it follows that $Z\oplus \Z$ is not cyclic.
\end{proof}

\begin{definition}
  A closed curve is essential if it is not null homotopic.
\end{definition}

\begin{proposition}
  Every essential closed curve in a compact hyperbolic surface $F$ is freely homotopic to a unique closed geodesic.
\end{proposition}
\begin{proof}
  Let $C$ be a closed curve in $F = \H/\Gamma$ and let $\tilde{C}$ be it's unique lift at fixed point $x$ in $\H$. Then the other end point of the lift is some $g x$, for some $g \in \Gamma$ since it's a loop when pushed down. Let $\tilde{\g}$ be the unique geodesic, which is the fixed axis of the hyperboic element $g$. 
\end{proof}
